%%%%%%%%%%%%%%%%%%%%%%%%%%%%%%%%%%%%%%%%%%%%%%%%%%%%%%%%%%%
\section{Conclusion and Future Work}\label{sec:conclusion}
%%%%%%%%%%%%%%%%%%%%%%%%%%%%%%%%%%%%%%%%%%%%%%%%%%%%%%%%%%%

This paper presented techniques for controlling the particle swarms in 1D and  2D grids.
These particles can be tracked and controlled by an external agent, but control inputs are applied uniformly so that each particle experiences the same applied forces. 
This paper presented benchmark algorithms for (1) mapping: building a representation of the free and obstacle regions of the workspace;
(2) foraging: ensuring at least one particle reaches each of a set of desired locations; 
and (3) coverage: ensuring every free region on the map is visited by at least one particle.
 These problems are inspired by challenges when using contrast particles for MR imaging.

These results form a baseline for future work, which should focus on improving performance. 
Extensions to 3D are especially relevant to the motivating problem of MR-scanning in living tissue.