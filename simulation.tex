
\section{Simulation}\label{sec:simulation}


\subsection{1D mapping}

  \begin{figure}
\begin{center}
	\includegraphics[width=1.0\columnwidth]{Prob1DcoverageGaps}
\end{center}
\caption{\label{fig:Prob1DcoverageGaps}
With a connected, 1D freespace $m$ = 1000 and $n=10$ particles, the distributions for the gap before the first $\overline{p}$ and after the last $\underline{p}$  gaps are symmetric. The maximum gap $\overline{g} \approx 250$.}
\end{figure}

\begin{figure}
\begin{center}
	\includegraphics[width=1.0\columnwidth]{SimReachBoundaryCoverage}
\end{center}
\caption{\label{fig:SimReachBoundaryCoverage}
Full coverage in 1D with $m$ = 1000 and $n=10$ particles requires 60.7 moves on average, while reaching the boundaries requires only 29.8.}
\end{figure}

  
 1D simulations were conducted in Mathematica, with code available at \cite{Arun2017GitHubUniformMapping}. 
 Fig.~\ref{fig:Prob1DcoverageGaps} shows the distributions for the minimum and maximum initial particle locations $\underline{p}$ and $\overline{p}$, the maximum gap $\overline{g}$, and the spread between the minimum and maximum $\overline{p}-\underline{p}$ for 1,000,000 Monte Carlo trials.
 The expected gap between the first particle and the boundary $\underline{p}$ is 90.94.
 The expected gap between the last particle and the boundary $\overline{p}$ is 90.98.
  The expected maximum gap is $\overline{g}$ is 273.9.
  
  Fig.~\ref{fig:SimReachBoundaryCoverage} shows that full coverage requires approximately twice the time required to explore the left and right boundaries when $m=1000$ and $n=100$.

  
  \subsection{2D mapping}  
2D simulations were conducted in Matlab, with code available at \cite{Arun2017matlabUniformMapping}. All simulations used maps with 5000 free cells.
  Each simulation trial was repeated 100 times.
  The number of particles ranged from 100 to 5000 by increments of 100.
  In each run except Fig.~\ref{fig:RegionvsFloodvsUniform} the particles were placed uniformly randomly throughout the workspace. 
  
 %%%%%%%%%%%%%%%  MappingAlg3maps
  The comparison plot Fig.~\ref{fig:MappingAlg3maps} between the mapping of  four 2D maps� \emph{H-tree}, \emph{complex}, \emph{empty rectangle} and \emph{linear} and a 1D map using the {\sc ClosestFrontier} algorithm, shows that the H-tree map requires the most moves because it has the highest number of turns. 
  In Fig.~\ref{fig:MappingAlg3maps} there is an observable difference in moves between the linear and rectangular workspaces. 
	One reason is because the perimeter of the linear map is much larger than the rectangular map. 
	The number of cells to explore is $3m+2 = 15,002$ for the linear map, but only $m + 2(50+100) = 5,300$ for the rectangular map. 
	Only when the number of particles is around 2/3 of the number of free spaces is there an overlap between the moves taken to map the rectangular space and the linear space. 

 
\begin{figure}
\begin{center}
	\includegraphics[width=1.0\columnwidth]{MappingAlg3maps}
\end{center}
\caption{\label{fig:MappingAlg3maps}
Comparison of mapping using the {\sc ClosestFrontier} algorithm on 2D maps of four types and 1D mapping on a linear map. 
Each map has 5000 free cells.
}
\end{figure}

 %%%%%%%%%%%%%%%  FrontierNodesVsk.pdf
The difference between algorithms is highlighted in Fig.~\ref{fig:FrontierNodesVsk}, which shows the number of frontier cells as a function of the number of moves commanded. All tests used $n=1000$ particles. {\sc ElectParticle} requires on average twice as many moves as {\sc ClosestFrontier} and {\sc RandomMoves} requires ten times as many moves as {\sc ClosestFrontier}.
The deviations for the {\sc ClosestFrontier} are also much less than the other two as seen from Fig.~\ref{fig:FrontierNodesVsk} and Fig.~\ref{fig:Alg_linlogplot}.
	
\begin{figure}
\begin{center}
	\includegraphics[width=1.0\columnwidth]{FrontierNodesVsklinlog.pdf}
\end{center}
\vspace{-2em}
\caption{\label{fig:FrontierNodesVsk}
Performing mapping on the complex 2D map with $n=1000$ particles. {\sc RandomMoves} requires 1683 moves, {\sc ElectParticle} requires 578 moves and {\sc ClosestFrontier} requires 215 moves on average}
\end{figure}	


%%%%%%%%%%%%%%%  fig:Alg_linlogplot

Fig.~\ref{fig:Alg_linlogplot} compares the performance of {\sc RandomMove}, {\sc ElectParticle}, and {\sc ClosestFrontier} on the complex 2D map. 
 For all algorithms the mean completion time and standard deviation of the completion time decreased with increasing numbers of particles. 
{\sc RandomMoves} performs worst, with the largest number of required moves and the largest standard deviation of required moves. {\sc RandomMoves} is slightly better than {\sc ElectParticle} for large numbers of particles, but both algorithms are beat by {\sc ClosestFrontier}, which has the minimum number of required moves and the smallest standard deviation.
  The maximum number of moves required using the {\sc ClosestFrontier} algorithm was for $k$=100 with an average of $\approx$1816 moves and standard deviation of 160 moves.
This reduces to four moves with 0 standard deviation when $n$= 5000 (the total number of free spaces).
%  The log plot shows that all three algorithms have an almost logarithmic relationship between $m$, the number of free spaces, and $k$, the number of particles.    % I don�t think we can claim this.  You'd have to fit a logrithmic fit to the data.  Try to do this with the linear-linear version of this data -- is it logarithmic?

\begin{figure}
\begin{center}
	\includegraphics[width=1.0\columnwidth]{Alg_linlogplot}
\end{center}
\caption{\label{fig:Alg_linlogplot}
Comparison of three algorithms - {\sc RandomMoves}, {\sc ElectParticle} and {\sc ClosestFrontier} for mapping the 2D Complex Map of 5000 free spaces.}
\end{figure}

%%%%%%%%%%%%%%%  fig:CoverageMappingForaging
Fig.~\ref{fig:CoverageMappingForaging} compares mapping, coverage, and subset coverage on the complex 2D map. All trials used {\sc ClosestFrontier}. Coverage is performed with a known map, but with all free cells initialized to be frontier cells. Similarly, subset coverage has a known map, but 10\% of the empty cells are labeled as frontier cells. Subset coverage is easier than coverage and coverage is easier than mapping.

\begin{figure}
\begin{center}
\vspace{-8em}
	\includegraphics[width=1.0\columnwidth]{CoverageMappingForaging.pdf}
\end{center}
\vspace{-9em}
\caption{\label{fig:CoverageMappingForaging}
Comparison of three related problems: mapping, coverage, and subset coverage on the complex 2D map.}
\end{figure}


%%%%%%%%%%%%%%%  fig:RegionvsFloodvsUniform
The final simulation test, shown in Fig.~\ref{fig:RegionvsFloodvsUniform}, compares the effect of different initial particle distributions in the complex 2D map.  
\emph{Region~fill} places all $n$ particles at a minimum Manhattan distance from a randomly selected location on the map.
\emph{Flood~fill} places one particle at a randomly selected location in the free space, and places the remaining particles according to a breadth-first expansion inside the free space.
\emph{Uniform~distribution} places the particles uniformly randomly.  
\emph{Region~fill} and \emph{flood~fill} have similar performance, while \emph{uniform~distribution} requires many fewer moves. 
This is because dispersing particles using only uniform inputs is difficult, and a \emph{uniform~distribution} starts with the particles dispersed, which allows it to map much faster.
\begin{figure}
\begin{center}
	\includegraphics[width=1.0\columnwidth]{RegionvsFloodvsUniform.pdf}
\end{center}
\caption{\label{fig:RegionvsFloodvsUniform}
Comparison with different distributions: \emph{flood~fill}, \emph{region~fill}, and \emph{uniform~distribution} for mapping on the complex 2D map. The results for \emph{flood~fill} and \emph{region~fill} overlap. }
\end{figure}








